\chapter{Bevezetés} % Introduction
\label{ch:intro}

A WebBeauty (továbbiakban WB) lehetőséget kínál a kisvállalkozók által gyártott termékek bemutatására és árusítására. Napjainkban megnőtt a kereslet a személyes webáruházak létrehozása iránt, ahol a kereskedők saját termékeiket kívánják értékesíteni, ezt pedig a WB megfelelően kiszolgálja. Az webalkalmazás nem csak a felhasználók számára könnyen kezelhető, hanem egyben a tulajdonosoknak is. Lehetőségük van arra, hogy egyszerűen menedzselhessék termékeiket és kapcsolatot tartsanak a lehetséges vásárlókkal.

Számomra a témaválasztás célja egy személyesen ismert vállalkozó megkeresésén alapult, aki szívesen értékesítené az általam készített webáruházon keresztül a termékeit. Az alapproblémát az vetette fel részemről, hogy egyénileg nem tudnám kiszolgálni az üzlettulajdonos által érkező folyamatos frissítési kéréseket. Ezt a felmerülő nehézséget úgy próbáltam megoldani, hogy a vállalkozó által is kényelmesen elérhetővé tettem azokat a funkciókat, ami a webalkalmazás aktualizáltságát biztosítja. Továbbá a program képes a tulajdonos és a vásárló közötti közvetlen kapcsolat látszatát kialakítani a chatbot\footnote{egy szoftver alkalmazás, aminek a támogatásával közvetlen emberi kapcsolat helyett egy virtuális ’asszisztenssel’ kommunikáljon.} funkció segítségével. Mivel ez egy egyszerűbb webalkalmazás, ezért nem egy teljesen egyénileg gondolkozó mesterséges intelligencia (MI)\footnote{sokféle megközelítést találhatunk a definícióját illetően. Személy szerint azt gondolom, hogy az MI egy tudatos gondolkozásra alkalmas, emberi beavatkozás nélkül cselekvőképes létforma, amit a legtöbbször számítástechnikai eszközökhöz/gépekhez társítunk.} alapú chatbotról esik szó, hanem egy adatbázisban tárolt, előre legenerált válaszokból álló szöveges adathalmazról beszélhetünk, ami kulcsszavas keresés segítségével működik.